% vim: expandtab tabstop=2 softtabstop=2 shiftwidth=2
\documentclass[a4paper,12pt]{article}
%\usepackage[scale=.8]{geometry}
\usepackage[english,noconfigs]{babel}
\usepackage{ifluatex,ifxetex}
\ifnum 0\ifxetex 1\fi\ifluatex 1\fi=0 % if pdftex
  \usepackage[utf8]{inputenc}
  \usepackage[T1]{fontenc}
\fi

\usepackage[ruled]{algorithm2e}
\usepackage[toc,page]{appendix}
\usepackage{minted}

\usepackage{graphicx}
\graphicspath{{graphics/}}

\usepackage{lipsum}

\title{%
  Universal predictor\\\small%
  Pattern-based prediction algorithms,\\%
  applications \& variations%
}

\author{%
  Carolina De Senne Garcia\\%
  Clément Durand%
}

\begin{document}
\maketitle

\vspace*{\fill}

\begin{abstract}
  \lipsum[1-2]
\end{abstract}

\vspace*{\fill}

\clearpage

\tableofcontents

\clearpage

\section*{Introduction}
\addcontentsline{toc}{section}{Introduction}

In this work we considered the problem of implementing an universal predictor based on patter matching. An universal predictor aims to perform well without knowing knowing the underlying probabilistic model of the data to be analysed.

In section \ref{paper_algos} we present an implementation of the basic idea of a predictor proposed by Ehrenfeucht and Mycielski \cite{basic_algo}, called \textit{Sampled Pattern Matching} (SPM) in \cite{paper}. Though the former was not an universal predictor, it was a good density estimator and was used by Jacquet et al. \cite{paper} as the base idea to develop an universal predictor. The implementation of this universal predictor is also presented in \ref{paper_algos}.

Section \ref{variation_algo} exposes a variation of the predictor algorithm proposed by us and based on Machine Learning techniques and pre-computation. We compare its performance with the universal predictor of section \ref{paper_algos}.

Finally, in section \ref{tests} we discuss performance tests for the algorithms implemented in this work, as well as implementation issues and possible solutions for them.

\section{Universal prediction algorithm}\label{paper_algos}

  
  \subsection{Simplified version}

    \begin{algorithm}
      \KwData{Character flow \textit{input}.}
      \KwResult{Predictions \textit{output} and success rate \textit{success}.}

      \caption{\label{simplified}Simplified version of universal prediction.}
    \end{algorithm}

  \subsection{Adding refinements}

    \begin{algorithm}
      \KwData{Character flow \textit{input}.}
      \KwResult{Predictions \textit{output} and success rate \textit{success}.}

      \caption{\label{simplified}Universal prediction.}
    \end{algorithm}

    \section{Variation proposal}\label{variation_algo}

  % Blabla: dans certains contextes (langues fixées, etc.) il peut être censé
  % d'avoir un algorithme qui apprend avant de faire de la prédiction.
  % Proposition d'un tel algorithme.

  \subsection{Learning algorithm}

  \subsection{Prediction algorithm}

  \subsection{Comparison}

  % Performance: difficilement comparable car on change l'équilibre du compromis
  % précomputation/prédiction.

  \section{Performance tryouts and issues}\label{tests}

  \subsection{Increasing the data sizes}

  \subsection{Distributing the algorithms}

\section*{Lessons learned}
\addcontentsline{toc}{section}{Lessons learned}

  \lipsum[8-9]

\clearpage
\begin{appendices}

  The algorithms presented in this report were implemented and evaluated
  in \emph{python3.5}. The code snippets below are simplified versions of
  the relevant parts of the implementation and are not $100\%$ correct for
  simplicity sake. Comments were added for clarity and the full source code
  remains available.

  \section{Simplified universal prediction}

    \inputminted[linenos]{python}{code/simplified.py}

  \clearpage
  \section{Universal prediction}

    \inputminted[linenos]{python}{code/complete.py}

  \clearpage
  \section{Learning and predicting}

    \subsection{Learning algorithm}

      \inputminted[linenos]{python}{code/learning.py}

    \clearpage
    \subsection{From substrings to predictions}

      \inputminted[linenos]{python}{code/predicting.py}

\end{appendices}

\bibliographystyle{plain}
\bibliography{references} 

\end{document}
